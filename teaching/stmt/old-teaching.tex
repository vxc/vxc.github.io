\documentclass[
%titlepage, 
%separate page for title
9pt, 
%twoside % page numbers on the outside
]{article}
%{report}
\usepackage[pdftex,dvipsnames,x11names,svgnames,table,fixpdftex,hyperref]{xcolor}
\usepackage{palatino}
\usepackage[square]{natbib}
\usepackage{array}
\usepackage{colortbl}
\usepackage{longtable}
\usepackage{dcolumn}
\usepackage{epigraph}
\usepackage{multirow}
\usepackage{graphicx}
%%% Comment the line below if you want a regular portrait
%%% style page layout 
% \include{pdfscreen}
\definecolor{lightgrey}{gray}{0.75}
%%% Comment the line below if you want a white background.
% \pagecolor{lightgrey}
\usepackage{palatino}
\usepackage[pdftitle={Proposal of Venkatesh Choppella's activities under Eureca Programme},%
colorlinks=true,% true
citecolor=blue,
linkcolor=Brown,
urlcolor=Brown,%navy%
]{hyperref}

\hypersetup{
pdfauthor= {Venkatesh Choppella <choppell@iiitmk.ac.in>},
pdfkeywords = {Research, Teaching, IIITM-K, Venkatesh Choppella, EURECA}
}

%                                  %%% pdfcreator, pdfproducer, 
%                                      and CreatioDate are automatically set
%                                      by pdflatex !!!
\pdfadjustspacing=1                %%% force LaTeX-like character spacing
%

\begin{document}
\title{Teaching Statement}
\author{Venkatesh Choppella}
\date{\today}
\maketitle

\begin{abstract}
  This document describes my efforts and plans for advancing
  computer science education.
\end{abstract}

\paragraph{Courses  past, present and future}
I have taught or assisted in the teaching of the following
courses: discrete mathematics, data structures, principles
of programming, programming languages, denotational
semantics, type systems, compilers, web technologies which I
am teaching this semester, databases, scientific computing,
and computational biology.  Web pages of some of those
courses are available online~\cite{choppella-courses}.  I
would like to teach the courses listed above and also design
new courses on program specification and verification, and
information security.

\paragraph{Abstraction Principles for Software Design}
My teaching follows \  --- but sometimes also leads --- \  my
research interests, which broadly span program language
design, implementation and abstraction principles for the
design of software.  Much of programming language research
is about semantics (ascribing meanings to programs) and
designing abstraction principles to improve the
organizational structure of programs.  Most application
programs are simple in terms of their formal algorithmic
complexity.  Instead, their complexity arises from the
structure of their implementations.  Their correctness,
efficiency and adaptibility in the face of competing and
cross cutting concerns like concurrency, distribution and
interaction makes programming hard.

\paragraph{Convergence} 
Today's software and computer systems exhibit a convergence
of concepts, methodologies and technologies and concepts
from science and engineering.  Education in computer science
should address how to harness this convergence.

% \paragraph{Formal notation}
% I am interested in developing pedagogy that introduces
% software engineers to formal notation at an early stage in
% their education.  Set theory, higher-order logic and
% axiomatic methods are eminently suited for providing precise
% formulations of many kinds of software requirements and
% design.  High level programming languages like Scheme and
% Haskell, and specification languages like Z and the
% Prototype Verification System (PVS) allow the programmer to
% use such formal notation as stylized equational logics.
% While the average mechanical or aerospace engineer will
% easily identify a rate equation, most programmers rarely
% even know, let alone use formal notations like sets,
% relations, and rewrite rules.  I plan to introduce formal
% notation early and often so programmers can think more like
% engineers.

\paragraph{ICASE}
My agenda for the next few years is to reformulate computer
science education along the ideas mentioned in the preceding
paragraphs, building in the process a programme for
Integrated Computing and Software engineering Education
(ICASE).  The emphasis on ICASE is to show how concepts
across various subdisciplines of computing are connected.
Theoretical foundations will be connected to models,
methodologies, technologies and applications.  ICASE will
systematically bring in the element of rigour in the
teaching of designing of software systems so that it matures
into an engineering discipline.

\paragraph{Six course suite}
ICASE will involve six courses: Principles of Programming,
Logic and Discrete Mathematics, Data Structures, Database
Design, Software Engineering and project management, and Web
Technologies.  These courses will be closely linked with
each other through assignments, homework problems and
projects that take a concept, and elaborate its various
aspects across different courses (e.g., first order logic
and its exposition in a discrete maths course, its
implementation in an interpreter in a programming course,
and its use as a query language in a database course).
ICASE could be offered as an undergraduate or master's level
stream.

\paragraph{Principles of Programming}
For the last two years, I have been involved in designing
and teaching a new course on the Principles of Programming
(PoP).  This course teaches fundamental programming concepts
like abstraction, recursion, closures, and persistence using
a definitional interpretation approach.  These concepts are
then used to build web 2.0 gadgets employing modern software
engineering practices of open source development.  This is a
more radical approach than that found in the landmark
programming texts~\cite{sicp,htdp,eopl}.  These other texts
do a brilliant job of teaching fundamentals, but do not
attempt to place programming concepts in today's computing
frameworks like the Web, mobile, and embedded systems.  In
PoP, students employ Scheme and Javascript with HTML and DOM
programming on a web browser.  Then they build a browser
based spreadsheet application ({\em \`a~la\/} Google
Spreadsheets) through a series of home work assignments.
Some of these points are elaborated in my recent talk at
Google R\&D, Benguluru~\cite{venk-google-talk-2008}.

\paragraph{PoP Book and Community}
My plan is to design, build and teach this PoP and its
variants (with embedded systems), and disseminate it using
the NPTEL model of video instruction.  The larger objective
is to build a learning community around an open wiki to
which teachers and students all over the world can
contribute.  I am working on an online, interactive book
tentatively titled {\em Principles of Programming with Web
  2.0\/} with my colleagues T B Dinesh and Guillaume
Marceau.

% \paragraph{Professional Development Programme}
% The ICASE initiative should be part of a larger professional
% development programme in computing, science and engineering.
% I would like to actively help IITH establish such a
% programme, which could use the the Stanford Centre for
% Professional Development~\cite{scpd} as its model.

\paragraph{Instructional Technologies and FOSS in Education}
I have been using free and open software (FOSS) since 1988,
and I will continue to actively promote its use in science
and engineering education, and general computing.  At
IIITM-K, my current employer, I introduced several
instructional technologies to make learning more effective:
wikis, online student course feedback, google calendar and
groups, version control, \LaTeX, and Emacs.  FOSS is also
making huge inroads into the world of scientific computing.
Pre-eminent institutes like the IIT's are specially
positioned well to promote free and open source software
(FOSS) through education.  To drive innovation in education,
IITH should work towards establishing an advanced centre for
instructional technologies and innovative learning.  IITH
could consider establishing this jointly with IGNOU and
other national institutes.  Ideally, a joint IIT-IGNOU
centre could be established at the Hyderabad campus.

% \paragraph{Conclusion: The romance of teaching}
% These are glorious times for the teaching profession.  We
% have a fantastic opportunity to make learning enjoyable,
% purposeful and open to a world wide classroom of lifelong
% learners.  But the challenges remain.  We need to revitalize
% education so that it enlightens and edifies, and remains
% holistic and useful.  Our education should show the student
% how to combine the abstract with the concrete, and lead them
% to understand the invariants driving a constantly changing
% world.  %
% % Can what we teach measure up to the famous
% % observation of Swami Vivekananda, ``Education is the
% % manifestation of the perfection already in Man.''?  
% The excitement of these challenges is what draws me to teaching.

% % We must innovate constantly to
% % remain relevant, because the student today has many choices?

\bibliographystyle{plain}

\bibliography{../biblio/venk,../biblio/rest}
\end{document}

